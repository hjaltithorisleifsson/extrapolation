\chapter{Initial Value Problems}

\section{The explicit midpoint rule}

Let \(f: \R\times \R^n \rightarrow \R^n\) be a smooth mapping and consider the initial value problem 
\begin{equation}\label{ivp}
y'(t) = f(t, y(t)), \quad y(a) = y_a, \quad t\in [a,b].
\end{equation}
The explicit midpoint methods is a method for computing an approximation to the solution of (\ref{ivp}), and it goes as follows: Let \(n\geq 1\) be an integer and \(h\coloneqq (b-a)/2n\). We then define recursively
\[
\xi_h(a) \coloneqq y_a, \quad \xi_h(a+h) \coloneqq \xi_h(a) + hf(a, \xi_h(a))
\]
and
\[
\xi_h(a + (i+1)h) \coloneqq \xi_h(a+(i-1)h) + 2hf(a+ih, \xi(a+ih)).
\]
Then \(\xi_h\) is an approximate solution to (\ref{ivp}) defined at \(a, a+h,\ldots ,b\). We are interested in the value \(X_f(h)\coloneqq \xi_h(b)\). It is possible to show that \(X_f(h)\) has an asymptotic expansion in \(h^2\). We have the following implementation in Python of the explicit midpoint rule for computing \(X_f(h)\).

\begin{minted}[tabsize=2, fontsize=\footnotesize]{python}
class ExplicitMidpointRule(Scheme):

	def __init__(self):
		super(ExplicitMidpointRule, self).__init__(2)

	def apply(self, ivp, n):
		h = (ivp.b - ivp.a) / (2 * n)
		y_sl = ivp.y0
		y_l = ivp.y0 + h * ivp.f(ivp.a, ivp.y0)

		for i in range(1, 2 * n):
			tmp = y_l
			y_l = y_sl + 2 * h * ivp.f(ivp.a + i * h, y_l)
			y_sl = tmp

		return y_l
\end{minted}

\section{Numerical experiments}

In this section we are going to extrapolate the explicit midpoint rule and analyze the convergence of the approximations as we extrapolate more often. Consider the initial value problem (\ref{ivp}). Let \(n_1 < n_2 < \cdots\) be some sequence of integers and \(h_i \coloneqq (b-a) / n_i\). Let \(X_{ij}\) the extrapolation table which we get from extrapolating in \(h^2\), using the points \((h_i,X_f(h_i))\). Let \(\varepsilon_i \coloneqq |X_{ii} - y(b)|\) be the absolute error. We are going to do the same convergence and efficency analysis as in the two previous chapters. We will both do the computations using high precision arithmetic with \(500\) correct digits and also in standard double precision.\\

%If \((a_n)\) is the sequence we use then the number of \(f\)-evaluations is %\(N_n = \sum_{k=1}^n 2\cdot a_n\). Thus for the harmonic sequence we have %\(N_n = n(n+1) \approx n^2\) for \(n\) large so if \(\varepsilon_n \sim %A\exp (-cN_n^q)\) then \(\varepsilon_n \sim A\exp(-cn^{2q})\) for large %\(n\). For the Romberg sequence then \(N_n = 2^{n+1} - 1\) so if %\(\varepsilon_n\sim A\exp(-cN_n^q)\) then \(\varepsilon_n \sim %A\exp(-c2^{nq})\) for large \(n\).\\

In those cases where we do not have an analytic solution to the equations, we computed a reference solution up to \(500\) significant digits. We did that by using extrapolation with the harmonic sequence and estimating the error as the difference between successive terms in the sequence of approximations.\\

Now we will consider the results of the experiments.

\subsection{Exponential growth}
First we will consider the following initial value problem:
\begin{equation}\label{42}
y'(x) = y(x),\quad y(0) = 0, \quad x\in [0,1]
\end{equation}
whose solution is the analytic function \(y(x) = e^x\).

\begin{figure}[H]
\centering
\begin{minipage}{0.45\textwidth}
\centering
\includegraphics[scale=0.45]{emr_plots/exp_growth.png}
\end{minipage}
\begin{minipage}{0.45\textwidth}
\centering
\includegraphics[scale=0.45]{emr_plots/exp_growth_hp_steps.png}
\end{minipage}
\end{figure}

\begin{figure}[H]
\centering
\begin{minipage}{0.45\textwidth}
\centering
\includegraphics[scale=0.45]{emr_plots/exp_growth_hp_log_log_pow_fit_trend.png}
\end{minipage}
\begin{minipage}{0.45\textwidth}
\centering
\includegraphics[scale=0.45]{emr_plots/exp_growth_hp_trend.png}
\end{minipage}
\end{figure}

\begin{table}[H]
    \centering
    \small
    \begin{tabular}{c|c||c|c|c|c|c|c}
Sequence & Plot & \(A\)-mean & \(A\)-var & \(c\)-mean & \(c\)-var & \(q\)-mean & \(q\)-var\\\hline
Romberg & lin-ln evals-error & \(3.929\cdot 10^{28}\) & \(3\) & \(34.22\) & \(0.08528\) & \(0.1755\) & \(0.02121\) \\
Harmonic & lin-ln evals-error & \(1.053\cdot 10^{12}\) & \(12.16\) & \(3.226\) & \(0.02133\) & \(0.6119\) & \(0.001013\) \\
Romberg & lin-ln steps-error & \(0.162\) & \(0.07703\) & \(0.8922\) & \(0.0008956\) & \(1.928\) & \(3.583\cdot 10^{-5}\) \\
Harmonic & lin-ln steps-error & \(2.469\cdot 10^{10}\) & \(11.7\) & \(3.279\) & \(0.01751\) & \(1.22\) & \(0.0007849\) \\
Romberg & ln-ln evals-error & \(1.232\) & \(0.1808\) & \(1.147\) & \(0.001975\) & \(2.075\) & \(8.865\cdot 10^{-5}\) \\
Harmonic & ln-ln evals-error & . & . & . & . & . & . \\
    \end{tabular}
    \label{tab:my_label}
\end{table}

Here we clearly have exponential convergence in the number of steps for the Romberg sequence but none of the models fits for the harmonic sequence.

\subsection{Logistic curve}

Then we will consider the following initial value problem
\begin{equation}\label{43}
y'(x) = y(x)(1-y(x)),\quad y(0) = 1/2, \quad x\in [0,1]
\end{equation}

whose solution is the sigmoid function
\[
\sigma(x) = \frac{1}{1 + e^{-x}}
\]
which is analytic.

\begin{figure}[H]
\centering
\begin{minipage}{0.45\textwidth}
\centering
\includegraphics[scale=0.45]{emr_plots/logistic.png}
\end{minipage}
\begin{minipage}{0.45\textwidth}
\centering
\includegraphics[scale=0.45]{emr_plots/logistic_hp_steps.png}
\end{minipage}
\end{figure}

\begin{figure}[H]
\centering
\begin{minipage}{0.45\textwidth}
\centering
\includegraphics[scale=0.45]{emr_plots/logistic_hp_log_log_pow_fit_trend.png}
\end{minipage}
\begin{minipage}{0.45\textwidth}
\centering
\includegraphics[scale=0.45]{emr_plots/logistic_hp_trend.png}
\end{minipage}
\end{figure}

\begin{table}[H]
    \centering
    \small
     \begin{tabular}{c|c||c|c|c|c|c|c}
Sequence & Plot & \(A\)-mean & \(A\)-var & \(c\)-mean & \(c\)-var & \(q\)-mean & \(q\)-var\\\hline
Romberg & lin-ln evals-error & \(7.183\cdot 10^{18}\) & \(2.999\) & \(26.25\) & \(0.08333\) & \(0.1874\) & \(0.02078\) \\
Harmonic & lin-ln evals-error & . & . & . & . & . & . \\
Romberg & lin-ln steps-error & \(9.544\cdot 10^{-5}\) & \(0.1065\) & \(0.6621\) & \(0.001969\) & \(1.995\) & \(7.267\cdot 10^{-5}\) \\
Harmonic & lin-ln steps-error & . & . & . & . & . & . \\
Romberg & ln-ln evals-error & \(0.0004803\) & \(0.1008\) & \(0.852\) & \(0.002435\) & \(2.15\) & \(0.0001118\) \\
Harmonic & ln-ln evals-error & . & . & . & . & . & . \\
    \end{tabular}
    \label{tab:my_label}
\end{table}

Here we clearly have exponential convergence in the number of steps for the Romberg sequence. The model does not fit in any case for the harmonic sequence.

\subsection{Tangens}

Now we will consider the following equation
\begin{equation}
y'(x) = 1 + y(x)^2, \quad y(0) = 0,\quad x\in [0,1]
\end{equation}

whose solution is 
\[
y(x) \coloneqq \tan(x)
\]
which is meromorphic and we are quite far from singularites.

\begin{figure}[H]
\centering
\begin{minipage}{0.45\textwidth}
\centering
\includegraphics[scale=0.45]{emr_plots/tangens.png}
\end{minipage}
\begin{minipage}{0.45\textwidth}
\centering
\includegraphics[scale=0.45]{emr_plots/tangens_hp_steps.png}
\end{minipage}
\end{figure}

\begin{figure}[H]
\centering
\begin{minipage}{0.45\textwidth}
\centering
\includegraphics[scale=0.45]{emr_plots/tangens_hp_log_log_pow_fit_trend.png}
\end{minipage}
\begin{minipage}{0.45\textwidth}
\centering
\includegraphics[scale=0.45]{emr_plots/tangens_hp_trend.png}
\end{minipage}
\end{figure}

\begin{table}[H]
    \centering
    \small
     \begin{tabular}{c|c||c|c|c|c|c|c}
Sequence & Plot & \(A\)-mean & \(A\)-var & \(c\)-mean & \(c\)-var & \(q\)-mean & \(q\)-var\\\hline
Romberg & lin-ln evals-error & \(1.351\cdot 10^{13}\) & \(2.997\) & \(13.1\) & \(0.1255\) & \(0.224\) & \(0.02402\) \\
Harmonic & lin-ln evals-error & \(616.1\) & \(0.9751\) & \(1.974\) & \(0.01426\) & \(0.517\) & \(0.001138\) \\
Romberg & lin-ln steps-error & \(0.2218\) & \(0.00522\) & \(0.2875\) & \(0.0001785\) & \(2.2\) & \(5.487\cdot 10^{-6}\) \\
Harmonic & lin-ln steps-error & \(204.6\) & \(0.9415\) & \(1.982\) & \(0.01258\) & \(1.033\) & \(0.0009701\) \\
Romberg & ln-ln evals-error & \(0.6649\) & \(0.2211\) & \(0.373\) & \(0.008112\) & \(2.379\) & \(0.0002882\) \\
Harmonic & ln-ln evals-error & . & . & . & . & . & . \\
    \end{tabular}
    \label{tab:my_label}
\end{table}

Here we clearly have exponential convergence in the number of steps for the Romberg sequence. None of the models fits well for the Harmonic sequence.

\subsection{Equation with singularity}

Now we will consider the following initial value problem:
\begin{equation}\label{46}
y'(t) = y^2(t),\quad y(0) = 1/(1+a), \quad t\in [0,1]
\end{equation}

whose solution is 
\[
y(t) = \frac{1}{1-(t-a)}.
\]
The solution is meromorphic with a pole at \(1+a\).
\begin{figure}[H]
\centering
\begin{minipage}{0.45\textwidth}
\centering
\includegraphics[scale=0.45]{emr_plots/singularity_0.png}
\end{minipage}
\begin{minipage}{0.45\textwidth}
\centering
\includegraphics[scale=0.45]{emr_plots/singularity_0_hp_steps.png}
\end{minipage}
\end{figure}

\begin{figure}[H]
\centering
\begin{minipage}{0.45\textwidth}
\centering
\includegraphics[scale=0.45]{emr_plots/singularity_0_hp_log_log_pow_fit_trend.png}
\end{minipage}
\begin{minipage}{0.45\textwidth}
\centering
\includegraphics[scale=0.45]{emr_plots/singularity_0_hp_trend.png}
\end{minipage}
\end{figure}

\begin{table}[H]
    \centering
    \small
     \begin{tabular}{c|c||c|c|c|c|c|c}
Sequence & Plot & \(A\)-mean & \(A\)-var & \(c\)-mean & \(c\)-var & \(q\)-mean & \(q\)-var\\\hline
Romberg & lin-ln evals-error & \(1.725\cdot 10^{16}\) & \(2.999\) & \(18.53\) & \(0.09702\) & \(0.2017\) & \(0.02064\) \\
Harmonic & lin-ln evals-error & \(728.6\) & \(0.9185\) & \(2.701\) & \(0.007961\) & \(0.5123\) & \(0.0006199\) \\
Romberg & lin-ln steps-error & \(0.05483\) & \(0.02945\) & \(0.4449\) & \(0.0008532\) & \(2.075\) & \(3.038\cdot 10^{-5}\) \\
Harmonic & lin-ln steps-error & \(169.4\) & \(0.8824\) & \(2.711\) & \(0.006852\) & \(1.024\) & \(0.0005169\) \\
Romberg & ln-ln evals-error & \(0.1965\) & \(0.09582\) & \(0.5737\) & \(0.002273\) & \(2.24\) & \(8.882\cdot 10^{-5}\) \\
Harmonic & ln-ln evals-error & . & . & . & . & . & . \\
    \end{tabular}
    \label{tab:my_label}
\end{table}

Here we seem to have exponential convergence in the number of steps for the Romberg sequence. For the Harmonic sequence, none of the models fits nicely.

\begin{figure}[H]
\centering
\begin{minipage}{0.45\textwidth}
\centering
\includegraphics[scale=0.45]{emr_plots/singularity_2.png}
\end{minipage}
\begin{minipage}{0.45\textwidth}
\centering
\includegraphics[scale=0.45]{emr_plots/singularity_2_hp_steps.png}
\end{minipage}
\end{figure}

\begin{figure}[H]
\centering
\begin{minipage}{0.45\textwidth}
\centering
\includegraphics[scale=0.45]{emr_plots/singularity_2_hp_log_log_pow_fit_trend.png}
\end{minipage}
\begin{minipage}{0.45\textwidth}
\centering
\includegraphics[scale=0.45]{emr_plots/singularity_2_hp_trend.png}
\end{minipage}
\end{figure}

\begin{table}[H]
    \centering
    \small
     \begin{tabular}{c|c||c|c|c|c|c|c}
Sequence & Plot & \(A\)-mean & \(A\)-var & \(c\)-mean & \(c\)-var & \(q\)-mean & \(q\)-var\\\hline
Romberg & lin-ln evals-error & \(567.8\) & \(1.143\) & \(0.1661\) & \(0.7022\) & \(0.5832\) & \(0.03661\) \\
Harmonic & lin-ln evals-error & \(241.1\) & \(0.4337\) & \(0.04452\) & \(0.2373\) & \(0.6456\) & \(0.007567\) \\
Romberg & lin-ln steps-error & \(108\) & \(0.05406\) & \(0.0004885\) & \(0.3319\) & \(4.297\) & \(0.002772\) \\
Harmonic & lin-ln steps-error & \(226.2\) & \(0.4359\) & \(-25.94\) & \(51.96\) & \(1.257\) & \(0.02703\) \\
Romberg & ln-ln evals-error & \(117.2\) & \(0.08201\) & \(0.0007221\) & \(0.4366\) & \(4.711\) & \(0.004126\) \\
Harmonic & ln-ln evals-error & . & . & . & . & . & . \\
    \end{tabular}
    \label{tab:my_label}
\end{table}

Here the model fits moderately well for Romberg sequence, when considering exponential convergence in the number of steps. We do not have nice fit for the harmonic sequence in any of the cases.

\begin{figure}[H]
\centering
\begin{minipage}{0.45\textwidth}
\centering
\includegraphics[scale=0.45]{emr_plots/singularity_4.png}
\end{minipage}
\begin{minipage}{0.45\textwidth}
\centering
\includegraphics[scale=0.45]{emr_plots/singularity_4_hp_steps.png}
\end{minipage}
\end{figure}

\begin{figure}[H]
\centering
\begin{minipage}{0.45\textwidth}
\centering
\includegraphics[scale=0.45]{emr_plots/singularity_4_hp_log_log_pow_fit_trend.png}
\end{minipage}
\begin{minipage}{0.45\textwidth}
\centering
\includegraphics[scale=0.45]{emr_plots/singularity_4_hp_trend.png}
\end{minipage}
\end{figure}

\begin{table}[H]
    \centering
    \small
     \begin{tabular}{c|c||c|c|c|c|c|c}
Sequence & Plot & \(A\)-mean & \(A\)-var & \(c\)-mean & \(c\)-var & \(q\)-mean & \(q\)-var\\\hline
Romberg & lin-ln evals-error & \(1.001\cdot 10^{4}\) & \(9.072\cdot 10^{-7}\) & \(9.09\cdot 10^{-5}\) & \(0.01109\) & \(0.9818\) & \(0.0001641\) \\
Harmonic & lin-ln evals-error & \(1\cdot 10^{4}\) & \(2.528\cdot 10^{-8}\) & \(0.0001699\) & \(0.0002823\) & \(0.7496\) & \(7.364\cdot 10^{-6}\) \\
Romberg & lin-ln steps-error & \(7433\) & \(0.3334\) & \(-67.64\) & \(3\) & \(5.423\) & \(0.3392\) \\
Harmonic & lin-ln steps-error & . & . & . & . & . & . \\
Romberg & ln-ln evals-error & \(4948\) & \(1\) & \(-198.5\) & \(1.232\) & \(4.103\) & \(1.002\) \\
Harmonic & ln-ln evals-error & . & . & . & . & . & . \\
    \end{tabular}
    \label{tab:my_label}
\end{table}

Here the model does not fit in any case.\\

The plot of \(q\) against \(a\) is as follows:

\begin{figure}[H]
\centering
\begin{minipage}{0.45\textwidth}
\centering
\includegraphics[scale=0.45]{emr_plots/log_p_vs_q_sing.png}
\end{minipage}
\end{figure}

\subsection{Equation with moderate singularity}

Now we will consider the following initial value problem
\begin{equation}
y'(t) = -\frac{1}{2y}, \quad y(0) = \sqrt{1+a},\quad t\in [0,1]\label{47}
\end{equation}
whose solution is 
\[
y(t) = \sqrt{1 - (t-a)}
\]
\begin{figure}[H]
\centering
\begin{minipage}{0.45\textwidth}
\centering
\includegraphics[scale=0.45]{emr_plots/quad_sing_0.png}
\end{minipage}
\begin{minipage}{0.45\textwidth}
\centering
\includegraphics[scale=0.45]{emr_plots/quad_sing_0_hp_steps.png}
\end{minipage}
\end{figure}

\begin{figure}[H]
\centering
\begin{minipage}{0.45\textwidth}
\centering
\includegraphics[scale=0.45]{emr_plots/quad_sing_0_hp_log_log_pow_fit_trend.png}
\end{minipage}
\begin{minipage}{0.45\textwidth}
\centering
\includegraphics[scale=0.45]{emr_plots/quad_sing_0_hp_trend.png}
\end{minipage}
\end{figure}

\begin{table}[H]
    \centering
    \small
     \begin{tabular}{c|c||c|c|c|c|c|c}
Sequence & Plot & \(A\)-mean & \(A\)-var & \(c\)-mean & \(c\)-var & \(q\)-mean & \(q\)-var\\\hline
Romberg & lin-ln evals-error & \(3.406\cdot 10^{16}\) & \(2.999\) & \(23.22\) & \(0.06365\) & \(0.1841\) & \(0.01443\) \\
Harmonic & lin-ln evals-error & \(0.4658\) & \(0.1195\) & \(3.123\) & \(0.0002015\) & \(0.4979\) & \(1.14\cdot 10^{-5}\) \\
Romberg & lin-ln steps-error & \(0.0008592\) & \(0.3997\) & \(0.5989\) & \(0.009262\) & \(1.979\) & \(0.0003485\) \\
Harmonic & lin-ln steps-error & \(0.1044\) & \(0.144\) & \(3.127\) & \(0.0002611\) & \(0.9956\) & \(1.552\cdot 10^{-5}\) \\
Romberg & ln-ln evals-error & \(0.002924\) & \(0.01955\) & \(0.7671\) & \(0.0003673\) & \(2.132\) & \(1.55\cdot 10^{-5}\) \\
Harmonic & ln-ln evals-error & . & . & . & . & . & . \\
    \end{tabular}
    \label{tab:my_label}
\end{table}

Here we clearly have exponential convergence in the number of steps and evaluations for the harmonic sequence. We also have it in the number of steps for the Romberg sequence.

\begin{figure}[H]
\centering
\begin{minipage}{0.45\textwidth}
\centering
\includegraphics[scale=0.45]{emr_plots/quad_sing_2.png}
\end{minipage}
\begin{minipage}{0.45\textwidth}
\centering
\includegraphics[scale=0.45]{emr_plots/quad_sing_2_hp_steps.png}
\end{minipage}
\end{figure}

\begin{figure}[H]
\centering
\begin{minipage}{0.45\textwidth}
\centering
\includegraphics[scale=0.45]{emr_plots/quad_sing_2_hp_log_log_pow_fit_trend.png}
\end{minipage}
\begin{minipage}{0.45\textwidth}
\centering
\includegraphics[scale=0.45]{emr_plots/quad_sing_2_hp_trend.png}
\end{minipage}
\end{figure}

\begin{table}[H]
    \centering
    \small
    \begin{tabular}{c|c||c|c|c|c|c|c}
Sequence & Plot & \(A\)-mean & \(A\)-var & \(c\)-mean & \(c\)-var & \(q\)-mean & \(q\)-var\\\hline
Romberg & lin-ln evals-error & \(1.977\) & \(1.277\) & \(0.912\) & \(0.1265\) & \(0.3841\) & \(0.009173\) \\
Harmonic & lin-ln evals-error & \(0.1185\) & \(0.9408\) & \(0.3049\) & \(0.2691\) & \(0.4676\) & \(0.007262\) \\
Romberg & lin-ln steps-error & \(0.05369\) & \(0.04044\) & \(0.009886\) & \(0.08471\) & \(3.134\) & \(0.001253\) \\
Harmonic & lin-ln steps-error & \(0.1007\) & \(0.6539\) & \(0.2999\) & \(0.2215\) & \(0.9368\) & \(0.006574\) \\
Romberg & ln-ln evals-error & \(0.06264\) & \(0.01892\) & \(0.01292\) & \(0.03735\) & \(3.418\) & \(0.0006328\) \\
Harmonic & ln-ln evals-error & . & . & . & . & . & . \\
    \end{tabular}
    \label{tab:my_label}
\end{table}

For the Romberg sequence, the model seems to fit moderately well when considering exponential convergence in the number of evaluations but it is not entirely clear since the model also fits moderately well when considering exponential converge in the number of steps and those two are mutually exclusive.\\

For the harmonic sequence we do not get nice fit in any case.

\begin{figure}[H]
\centering
\begin{minipage}{0.45\textwidth}
\centering
\includegraphics[scale=0.45]{emr_plots/quad_sing_4.png}
\end{minipage}
\begin{minipage}{0.45\textwidth}
\centering
\includegraphics[scale=0.45]{emr_plots/quad_sing_4_hp_steps.png}
\end{minipage}
\end{figure}

\begin{figure}[H]
\centering
\begin{minipage}{0.45\textwidth}
\centering
\includegraphics[scale=0.45]{emr_plots/quad_sing_4_hp_log_log_pow_fit_trend.png}
\end{minipage}
\begin{minipage}{0.45\textwidth}
\centering
\includegraphics[scale=0.45]{emr_plots/quad_sing_4_hp_trend.png}
\end{minipage}
\end{figure}

\begin{table}[H]
    \centering
    \small
     \begin{tabular}{c|c||c|c|c|c|c|c}
Sequence & Plot & \(A\)-mean & \(A\)-var & \(c\)-mean & \(c\)-var & \(q\)-mean & \(q\)-var\\\hline
Romberg & lin-ln evals-error & \(1.59\) & \(1.047\) & \(1.341\) & \(0.3366\) & \(0.1989\) & \(0.08076\) \\
Harmonic & lin-ln evals-error & . & . & . & . & . & . \\
Romberg & lin-ln steps-error & \(0.1375\) & \(0.1919\) & \(0.04595\) & \(0.8511\) & \(2.117\) & \(0.05595\) \\
Harmonic & lin-ln steps-error & \(10.05\) & \(18.03\) & \(1.442\) & \(1.062\) & \(0.3442\) & \(0.1232\) \\
Romberg & ln-ln evals-error & \(0.148\) & \(0.1921\) & \(0.05535\) & \(0.7589\) & \(2.281\) & \(0.05284\) \\
Harmonic & ln-ln evals-error & . & . & . & . & . & . \\
    \end{tabular}
    \label{tab:my_label}
\end{table}

Here, we do not have any clear fit.

The plot of \(q\) against \(a\) is as follows:

\begin{figure}[H]
\centering
\begin{minipage}{0.45\textwidth}
\centering
\includegraphics[scale=0.45]{emr_plots/log_p_vs_q_quad_sing.png}
\end{minipage}
\end{figure}

\subsection{Circular rotation}

Now we will consider the following system of equations:
\begin{equation}\label{48}
(y_1(t),y_2(t))' = (-y_2(t), y_1(t)), \quad y(0) = (1,0), \quad t\in [0,\pi /2]
\end{equation}
whose solution is 
\[
(y_1(t),y_2(t)) = (\cos t, \sin t)
\]
which is entire.

\begin{figure}[H]
\centering
\begin{minipage}{0.45\textwidth}
\centering
\includegraphics[scale=0.45]{emr_plots/rotation.png}
\end{minipage}
\begin{minipage}{0.45\textwidth}
\centering
\includegraphics[scale=0.45]{emr_plots/rotation_hp_steps.png}
\end{minipage}
\end{figure}

\begin{figure}[H]
\centering
\begin{minipage}{0.45\textwidth}
\centering
\includegraphics[scale=0.45]{emr_plots/rotation_hp_log_log_pow_fit_trend.png}
\end{minipage}
\begin{minipage}{0.45\textwidth}
\centering
\includegraphics[scale=0.45]{emr_plots/rotation_hp_trend.png}
\end{minipage}
\end{figure}

\begin{table}[H]
    \centering
    \small
     \begin{tabular}{c|c||c|c|c|c|c|c}
Sequence & Plot & \(A\)-mean & \(A\)-var & \(c\)-mean & \(c\)-var & \(q\)-mean & \(q\)-var\\\hline
Romberg & lin-ln evals-error & \(9.158\cdot 10^{24}\) & \(3\) & \(27.06\) & \(0.103\) & \(0.1896\) & \(0.02387\) \\
Harmonic & lin-ln evals-error & \(1.269\cdot 10^{13}\) & \(12.98\) & \(2.597\) & \(0.0327\) & \(0.6252\) & \(0.001654\) \\
Romberg & lin-ln steps-error & \(1.09\) & \(0.0003639\) & \(0.6727\) & \(6.223\cdot 10^{-6}\) & \(2.007\) & \(2.397\cdot 10^{-7}\) \\
Harmonic & lin-ln steps-error & \(4.499\cdot 10^{11}\) & \(12.52\) & \(2.643\) & \(0.02762\) & \(1.246\) & \(0.001318\) \\
Romberg & ln-ln evals-error & \(7.126\) & \(0.3484\) & \(0.8677\) & \(0.005226\) & \(2.163\) & \(0.0002203\) \\
Harmonic & ln-ln evals-error & . & . & . & . & . & . \\
    \end{tabular}
    \label{tab:my_label}
\end{table}

The harmonic sequence works better then Romberg and we get down to machine level precision using either sequence when using standard floating point arithmetic.\\

We clearly have exponential convergence in the number of steps for the Romberg sequence but we do not have any nice fit for the harmonic sequence.

\subsection{Mathematical pendulum}

Now we will consider the mathematical pendulum equation:
\begin{equation}
y''(t) + \sin y(t) = 0,\quad y(0) = 0,\, y'(0) = 1, \quad t\in [0,1].
\end{equation}

\begin{figure}[H]
\centering
\begin{minipage}{0.45\textwidth}
\centering
\includegraphics[scale=0.45]{emr_plots/oscillation.png}
\end{minipage}
\begin{minipage}{0.45\textwidth}
\centering
\includegraphics[scale=0.45]{emr_plots/oscillation_hp_steps.png}
\end{minipage}
\end{figure}

\begin{figure}[H]
\centering
\begin{minipage}{0.45\textwidth}
\centering
\includegraphics[scale=0.45]{emr_plots/oscillation_hp_log_log_pow_fit_trend.png}
\end{minipage}
\begin{minipage}{0.45\textwidth}
\centering
\includegraphics[scale=0.45]{emr_plots/oscillation_hp_trend.png}
\end{minipage}
\end{figure}

\begin{table}[H]
    \centering
    \small
     \begin{tabular}{c|c||c|c|c|c|c|c}
Sequence & Plot & \(A\)-mean & \(A\)-var & \(c\)-mean & \(c\)-var & \(q\)-mean & \(q\)-var\\\hline
Romberg & lin-ln evals-error & \(4.852\cdot 10^{18}\) & \(3\) & \(23.49\) & \(0.06226\) & \(0.1898\) & \(0.01102\) \\
Harmonic & lin-ln evals-error & . & . & . & . & . & . \\
Romberg & lin-ln steps-error & \(0.01054\) & \(1.671\) & \(0.603\) & \(0.03526\) & \(2.013\) & \(0.001162\) \\
Harmonic & lin-ln steps-error & . & . & . & . & . & . \\
Romberg & ln-ln evals-error & \(0.0311\) & \(0.7449\) & \(0.7694\) & \(0.01405\) & \(2.17\) & \(0.0005522\) \\
Harmonic & ln-ln evals-error & . & . & . & . & . & . \\
    \end{tabular}
    \label{tab:my_label}
\end{table}

Here we seem to have exponential convergence in the number of steps for the Romberg sequence. The harmonic sequence works is faster and we almost get down to machine level precision using either sequence. 

\subsection{Federpendel}

Now we will consider the equation of motion for das Federpendel or the spring pendulum:
\[
\bfp' = -(|\bfq| - 1)\frac{\bfq}{|\bfq|} - {1\choose 0}, \quad \bfq' = \bfp
\]
where \(\bfp\) and \(\bfq\) are two dimensional vectors. We will consider it with the initial condition \(\bfq(0) = (1,0)\) and \(\bfp(0) = (0,1)\) and try to both estimate the solution at time \(t = 1\) and time \(t = 2\).

\begin{figure}[H]
\centering
\begin{minipage}{0.45\textwidth}
\centering
\includegraphics[scale=0.45]{emr_plots/federpendel.png}
\end{minipage}
\begin{minipage}{0.45\textwidth}
\centering
\includegraphics[scale=0.45]{emr_plots/federpendel_1_hp_steps.png}
\end{minipage}
\end{figure}

\begin{figure}[H]
\centering
\begin{minipage}{0.45\textwidth}
\centering
\includegraphics[scale=0.45]{emr_plots/federpendel_1_hp_log_log_pow_fit_trend.png}
\end{minipage}
\begin{minipage}{0.45\textwidth}
\centering
\includegraphics[scale=0.45]{emr_plots/federpendel_1_hp_trend.png}
\end{minipage}
\end{figure}

\begin{table}[H]
    \centering
    \small
     \begin{tabular}{c|c||c|c|c|c|c|c}
Sequence & Plot & \(A\)-mean & \(A\)-var & \(c\)-mean & \(c\)-var & \(q\)-mean & \(q\)-var\\\hline
Romberg & lin-ln evals-error & \(2.325^{15}\) & \(2.897\) & \(20.8\) & \(0.07387\) & \(0.1913\) & \(0.01729\) \\
Harmonic & lin-ln evals-error & . & . & . & . & . & . \\
Romberg & lin-ln steps-error & \(0.006769\) & \(0.2266\) & \(0.5219\) & \(0.007752\) & \(2.019\) & \(0.0002881\) \\
Harmonic & lin-ln steps-error & . & . & . & . & . & . \\
Romberg & ln-ln evals-error & \(0.02348\) & \(0.2334\) & \(0.6702\) & \(0.003067\) & \(2.176\) & \(0.000111\) \\
Harmonic & ln-ln evals-error & . & . & . & . & . & . \\
    \end{tabular}
    \label{tab:my_label}
\end{table}

Here we seem to have exponential convergence in the number of steps for the Romberg sequence. None of the models fits well for the harmonic sequence.\\

The harmonic sequence performes better we almost get down to machine level precision using either sequence, in double precision floating point arithmetic. 

\begin{figure}[H]
\centering
\begin{minipage}{0.45\textwidth}
\centering
\includegraphics[scale=0.45]{emr_plots/federpendel_2.png}
\end{minipage}
\begin{minipage}{0.45\textwidth}
\centering
\includegraphics[scale=0.45]{emr_plots/federpendel_2_hp_steps.png}
\end{minipage}
\end{figure}

\begin{figure}[H]
\centering
\begin{minipage}{0.45\textwidth}
\centering
\includegraphics[scale=0.45]{emr_plots/federpendel_2_hp_log_log_pow_fit_trend.png}
\end{minipage}
\begin{minipage}{0.45\textwidth}
\centering
\includegraphics[scale=0.45]{emr_plots/federpendel_2_hp_trend.png}
\end{minipage}
\end{figure}

\begin{table}[H]
    \centering
    \small
     \begin{tabular}{c|c||c|c|c|c|c|c}
Sequence & Plot & \(A\)-mean & \(A\)-var & \(c\)-mean & \(c\)-var & \(q\)-mean & \(q\)-var\\\hline
Romberg & lin-ln evals-error & \(2.562\cdot 10^{13}\) & \(3\) & \(13.43\) & \(0.1715\) & \(0.224\) & \(0.04363\) \\
Harmonic & lin-ln evals-error & . & . & . & . & . & . \\
Romberg & lin-ln steps-error & \(0.07725\) & \(0.4469\) & \(0.2919\) & \(0.04082\) & \(2.194\) & \(0.001626\) \\
Harmonic & lin-ln steps-error & . & . & . & . & . & . \\
Romberg & ln-ln evals-error & \(0.2609\) & \(0.6696\) & \(0.3816\) & \(0.05891\) & \(2.373\) & \(0.002839\) \\
Harmonic & ln-ln evals-error & . & . & . & . & . & . \\
    \end{tabular}
    \label{tab:my_label}
\end{table}

Here the model for exponential convergence in the number of steps seems to fit moderately well for the Romberg sequence but none of the models fits for the harmonic sequence.\\

The harmonic sequence works better and we get almost down to machine level precision in standard double precision floating point arithmetic.

\subsection{Lorenz equations}

The Lorenz equations are the following system: 
\[
\frac{dx}{dt} = \sigma (y-x),\quad \frac{dy}{dt} = x(\rho - z) - y,\quad \frac{dz}{dt} = xy - \beta z
\]
where \(\sigma,\,\rho\) and \(\beta\) are constants. In our experiment, the constants are set to \(\sigma = 10\), \(\rho = 28\) and \(\beta = 8/3\). The initial condition we will consider is \((x(0),y(0),z(0)) = (1,1,1)\).\\

\begin{figure}[H]
\centering
\begin{minipage}{0.45\textwidth}
\centering
\includegraphics[scale=0.45]{emr_plots/lorenz.png}
\end{minipage}
\begin{minipage}{0.45\textwidth}
\centering
\includegraphics[scale=0.45]{emr_plots/lorenz_hp_steps.png}
\end{minipage}
\end{figure}

\begin{figure}[H]
\centering
\begin{minipage}{0.45\textwidth}
\centering
\includegraphics[scale=0.45]{emr_plots/lorenz_hp_log_log_pow_fit_trend.png}
\end{minipage}
\begin{minipage}{0.45\textwidth}
\centering
\includegraphics[scale=0.45]{emr_plots/lorenz_hp_trend.png}
\end{minipage}
\end{figure}

\begin{table}[H]
    \centering\small
     \begin{tabular}{c|c||c|c|c|c|c|c}
Sequence & Plot & \(A\)-mean & \(A\)-var & \(c\)-mean & \(c\)-var & \(q\)-mean & \(q\)-var\\\hline
Romberg & lin-ln evals-error & \(3.447\cdot 10^{21}\) & \(3\) & \(21.2\) & \(0.1242\) & \(0.2061\) & \(0.02692\) \\
Harmonic & lin-ln evals-error & . & . & . & . & . & . \\
Romberg & lin-ln steps-error & \(2.886\) & \(0.04035\) & \(0.4956\) & \(0.0009672\) & \(2.098\) & \(3.499\cdot 10^{-5}\) \\
Harmonic & lin-ln steps-error & . & . & . & . & . & . \\
Romberg & ln-ln evals-error & \(15.97\) & \(0.4928\) & \(0.6421\) & \(0.01116\) & \(2.265\) & \(0.0004469\) \\
Harmonic & ln-ln evals-error & . & . & . & . & . & . \\
    \end{tabular}
    \label{tab:my_label}
\end{table}

Here the harmonic sequence works better and we get down to machine level precision in standard double precision arithmetic.\\

The model for exponential convergence in the number of steps seems to fit well for the Romberg sequence but none of the models fits for the harmonic sequence.

\begin{figure}[H]
\centering
\begin{minipage}{0.45\textwidth}
\centering
\includegraphics[scale=0.45]{emr_plots/lorenz_02.png}
\end{minipage}
\begin{minipage}{0.45\textwidth}
\centering
\includegraphics[scale=0.45]{emr_plots/lorenz_02_hp_steps.png}
\end{minipage}
\end{figure}

\begin{figure}[H]
\centering
\begin{minipage}{0.45\textwidth}
\centering
\includegraphics[scale=0.45]{emr_plots/lorenz_02_hp_log_log_pow_fit_trend.png}
\end{minipage}
\begin{minipage}{0.45\textwidth}
\centering
\includegraphics[scale=0.45]{emr_plots/lorenz_02_hp_trend.png}
\end{minipage}
\end{figure}

\begin{table}[H]
    \centering
    \small
    \begin{tabular}{c|c||c|c|c|c|c|c}
Sequence & Plot & \(A\)-mean & \(A\)-var & \(c\)-mean & \(c\)-var & \(q\)-mean & \(q\)-var\\\hline
Romberg & lin-ln evals-error & \(1.544\cdot 10^{17}\) & \(3\) & \(12.97\) & \(0.1833\) & \(0.2372\) & \(0.03499\) \\
Harmonic & lin-ln evals-error & . & . & . & . & . & . \\
Romberg & lin-ln steps-error & \(129.1\) & \(0.3322\) & \(0.2655\) & \(0.01477\) & \(2.27\) & \(0.0004755\) \\
Harmonic & lin-ln steps-error & . & . & . & . & . & . \\
Romberg & ln-ln evals-error & \(506.5\) & \(0.8695\) & \(0.3478\) & \(0.03817\) & \(2.457\) & \(0.00137\) \\
Harmonic & ln-ln evals-error & . & . & . & . & . & . \\
    \end{tabular}
    \label{tab:my_label}
\end{table}


Here we also get down to machine level precision in standard double precision floating point arithmetic. The harmonic sequence performes better.\\

None of the models fits for the harmonic sequence but we seem to have exponential convergence in the number of steps for the romberg sequence.

The values of the optimal parameters in the fitting of the number evaluations against the error, are:

\begin{table}[H]
    \centering
    \begin{tabular}{c|c||c|c|c}
           IVP & Sequence & \(b\) & \(c\) & \(q\) \\\hline\hline
$y'=y$, $y(0) = 0$ & Romberg & \(27.187\) & \(22.015\) & \(0.20227\) \\
$y'=y$, $y(0) = 0$ & Harmonic & \(8.8124\) & \(3.0433\) & \(0.61436\) \\
$y' = y(1-y)$ & Romberg & \(17.604\) & \(19.237\) & \(0.20418\) \\
$y' = y(1-y)$ & Harmonic & \(1.7927\) & \(4.0933\) & \(0.51064\) \\
$y' = 1 + y^2$, $y(0) = 0$ & Romberg & \(11.465\) & \(8.8186\) & \(0.24549\) \\
$y' = 1 + y^2$, $y(0) = 0$ & Harmonic & \(3.3496\) & \(1.8647\) & \(0.51993\) \\
$(y_1,y_2)' = (-y_2,y_1)$, $y(0) = (1,0)$ & Romberg & \(22.626\) & \(16.846\) & \(0.2184\) \\
$(y_1,y_2)' = (-y_2,y_1)$, $y(0) = (1,0)$ & Harmonic & \(9.7592\) & \(2.4171\) & \(0.62765\) \\
$y' = y^2$, $y(0) = 1/2$ & Romberg & \(14.78\) & \(12.517\) & \(0.22455\) \\
$y' = y^2$, $y(0) = 1/2$ & Harmonic & \(3.5994\) & \(2.5894\) & \(0.51472\) \\
$y'=y^2$, $y(0) = 1/(1+10^{-2})$ & Romberg & \(5.3983\) & \(0.1961\) & \(0.4989\) \\
$y'=y^2$, $y(0) = 1/(1+10^{-2})$ & Harmonic & \(4.7796\) & \(0.032416\) & \(0.6592\) \\
$y'=y^2$, $y(0) = 1/(1+10^{-4})$ & Romberg & \(9.212\) & \(0.00012308\) & \(0.94461\) \\
$y'=y^2$, $y(0) = 1/(1+10^{-4})$ & Harmonic & \(9.2104\) & \(0.00016925\) & \(0.74975\) \\
$y' = -1/2y$, $y(0) = \sqrt{2}$ & Romberg & \(17.142\) & \(17.376\) & \(0.2007\) \\
$y' = -1/2y$, $y(0) = \sqrt{2}$ & Harmonic & \(-0.45512\) & \(3.1436\) & \(0.49744\) \\
$y' = -1/2y$, $y(0) = \sqrt{1+10^{-2}}$ & Romberg & \(0.063222\) & \(1.0479\) & \(0.35645\) \\
$y' = -1/2y$, $y(0) = \sqrt{1+10^{-2}}$ & Harmonic & \(-1.7425\) & \(0.33371\) & \(0.45544\) \\
$y' = -1/2y$, $y(0) = \sqrt{1+10^{-4}}$ & Romberg & \(0.13652\) & \(1.2653\) & \(0.19436\) \\
$y' = -1/2y$, $y(0) = \sqrt{1+10^{-4}}$ & Harmonic & \(2.2343\) & \(3.3206\) & \(0.091009\) \\
$y'' + \sin(y) = 0$ & Romberg & \(22.275\) & \(19.017\) & \(0.20078\) \\
$y'' + \sin(y) = 0$ & Harmonic & \(4.7064\) & \(3.6116\) & \(0.50678\) \\
Federpendel, estimate after 1 time unit. & Romberg & \(17.532\) & \(15.778\) & \(0.20624\) \\
Federpendel, estimate after 1 time unit. & Harmonic & \(0.30737\) & \(2.706\) & \(0.49211\) \\
Federpendel, estimate after 2 time units. & Romberg & \(14.66\) & \(11.217\) & \(0.22272\) \\
Federpendel, estimate after 2 time units. & Harmonic & \(0.79006\) & \(1.789\) & \(0.48379\) \\
Lorenz, estimate after 0.1 time steps. & Romberg & \(18.654\) & \(12.939\) & \(0.2354\) \\
Lorenz, estimate after 0.1 time steps. & Harmonic & \(12.615\) & \(4.0068\) & \(0.5111\) \\
Lorenz, estimate after 0.2 time steps. & Romberg & \(14.264\) & \(7.2173\) & \(0.27063\) \\
Lorenz, estimate after 0.2 time steps. & Harmonic & \(13.212\) & \(2.7376\) & \(0.50732\) \\
    \end{tabular}
    \caption{Optimal parameters by test case}
    \label{tab:my_label}
\end{table}

We note that in those cases where the singularities of the solutions are not very close to our time interval, then \(q\) is close to \(0.5\) for the harmonic sequence and close to \(0.2\) for the Romberg sequence.\\

The values of the optimal parameters in the fitting of the number of extrapolation steps against the error, are:

\begin{table}[H]
    \centering
    \begin{tabular}{c|c||c|c|c}
           IVP & Sequence & \(b\) & \(c\) & \(q\) \\\hline\hline
$y'=y$, $y(0) = 0$ & Romberg & \(-1.0083\) & \(0.9557\) & \(1.904\) \\
$y'=y$, $y(0) = 0$ & Harmonic & \(6.425\) & \(3.1061\) & \(1.225\) \\
$y' = y(1-y)$ & Romberg & \(-7.1901\) & \(0.82672\) & \(1.916\) \\
$y' = y(1-y)$ & Harmonic & \(-0.19572\) & \(4.1127\) & \(1.0204\) \\
$y' = 1 + y^2$, $y(0) = 0$ & Romberg & \(-1.2103\) & \(0.30542\) & \(2.1771\) \\
$y' = 1 + y^2$, $y(0) = 0$ & Harmonic & \(2.405\) & \(1.8763\) & \(1.0387\) \\
$(y_1,y_2)' = (-y_2,y_1)$, $y(0) = (1,0)$ & Romberg & \(0.15924\) & \(0.67746\) & \(2.004\) \\
$(y_1,y_2)' = (-y_2,y_1)$, $y(0) = (1,0)$ & Harmonic & \(7.7307\) & \(2.4712\) & \(1.2513\) \\
$y' = y^2$, $y(0) = 1/2$ & Romberg & \(-2.193\) & \(0.48799\) & \(2.0424\) \\
$y' = y^2$, $y(0) = 1/2$ & Harmonic & \(2.3181\) & \(2.6033\) & \(1.0284\) \\
$y'=y^2$, $y(0) = 1/(1+10^{-2})$ & Romberg & \(4.6269\) & \(0.00074557\) & \(4.042\) \\
$y'=y^2$, $y(0) = 1/(1+10^{-2})$ & Harmonic & \(4.7474\) & \(0.033263\) & \(1.3137\) \\
$y'=y^2$, $y(0) = 1/(1+10^{-4})$ & Romberg & \(-2507.7\) & \(-2517.1\) & \(-0.00010114\) \\
$y'=y^2$, $y(0) = 1/(1+10^{-4})$ & Harmonic & \(9.2101\) & \(0.00017545\) & \(1.4929\) \\
$y' = -1/2y$, $y(0) = \sqrt{2}$ & Romberg & \(-5.0269\) & \(0.76221\) & \(1.8929\) \\
$y' = -1/2y$, $y(0) = \sqrt{2}$ & Harmonic & \(-1.8945\) & \(3.1518\) & \(0.99437\) \\
$y' = -1/2y$, $y(0) = \sqrt{1+10^{-2}}$ & Romberg & \(-2.1638\) & \(0.015532\) & \(2.9565\) \\
$y' = -1/2y$, $y(0) = \sqrt{1+10^{-2}}$ & Harmonic & \(-1.8689\) & \(0.33241\) & \(0.91148\) \\
$y' = -1/2y$, $y(0) = \sqrt{1+10^{-4}}$ & Romberg & \(-1.4291\) & \(0.062542\) & \(1.8173\) \\
$y' = -1/2y$, $y(0) = \sqrt{1+10^{-4}}$ & Harmonic & \(1.3726\) & \(2.6336\) & \(0.20682\) \\
$y'' + \sin(y) = 0$ & Romberg & \(-1.9686\) & \(0.83745\) & \(1.8917\) \\
$y'' + \sin(y) = 0$ & Harmonic & \(2.9822\) & \(3.6265\) & \(1.0128\) \\
Federpendel, estimate after 1 time unit. & Romberg & \(-2.8486\) & \(0.67856\) & \(1.9249\) \\
Federpendel, estimate after 1 time unit. & Harmonic & \(-0.90208\) & \(2.7108\) & \(0.98385\) \\
Federpendel, estimate after 2 time units. & Romberg & \(-0.45874\) & \(0.44537\) & \(2.0272\) \\
Federpendel, estimate after 2 time units. & Harmonic & \(0.020877\) & \(1.79\) & \(0.96743\) \\
Lorenz, estimate after 0.1 time steps. & Romberg & \(0.59151\) & \(0.47509\) & \(2.112\) \\
Lorenz, estimate after 0.1 time steps. & Harmonic & \(10.659\) & \(4.0254\) & \(1.0213\) \\
Lorenz, estimate after 0.2 time steps. & Romberg & \(3.1116\) & \(0.21429\) & \(2.3417\) \\
Lorenz, estimate after 0.2 time steps. & Harmonic & \(11.891\) & \(2.7477\) & \(1.014\) \\
    \end{tabular}
    \caption{Optimal parameters by test case}
    \label{tab:my_label}
\end{table}